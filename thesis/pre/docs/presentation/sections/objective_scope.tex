\section{Objective and Scope}

\begin{frame}{Objective}
    \begin{enumerate}
        \item Comprehensively review self-correcting LLMs for software debugging.

        \item Investigate instructing LLMs to optimally resolve bugs, improving on published methods.

        \item Identify evaluation benchmarks.

        \item Implement a Chain-of-Thought prompting and modular code analysis model, and researching improvements.

        \item Comparatively evaluate results against published methods.
    \end{enumerate}
\end{frame}

\begin{frame}{Scope}
    \begin{enumerate}
        \item Focus on self-correction techniques in LLMs for Python data science error debugging.

        \item Use the DS-1000~\cite{pmlr-v202-lai23b} dataset of Python data science bugs.

        \item Utilize Azure OpenAI ChatGPT Turbo 3.5 LLM.

        \item Develop a Chain-of-Thought prompting and modular code analysis system to effectively resolve bugs.
    \end{enumerate}
\end{frame}

\begin{frame}{Dataset}
    1000 Python data science coding problems, each with:
    \begin{block}{Problem Description}
        \small
        Problem:
        How do I get the dimensions of an array? For instance, this is (2, 2):\\
        a = np.array([[1,2],[3,4]])
    \end{block}

    \begin{columns}[T]
        \begin{column}{0.50\textwidth}
            \begin{block}{Buggy Code}
                \small
                <code>\\
                import numpy as np\\
                a = np.array([[1,2],[3,4]])\\
                </code>\\
                result = $\ldots$ \# put solution in this variable\\
                BEGIN SOLUTION\\
                <code>
            \end{block}
        \end{column}
        \begin{column}{0.45\textwidth}
            \begin{block}{Unit Test}
                \small
                def test(result, ans):\\
                \ \ \ \ assert\_array\_equal(result, ans)\\
                \ \ \ \ return 1
            \end{block}
        \end{column}
    \end{columns}
\end{frame}
