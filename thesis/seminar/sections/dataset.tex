\section{Benchmark Dataset}

\begin{frame}{DS-1000}
    DS-1000~\cite{pmlr-v202-lai23b}: 1000 Python data science coding problems, each with:
    \begin{block}{Problem Description}
        \small
        Problem:
        How do I get the dimensions of an array? For instance, this is (2, 2):\\
        a = np.array([[1,2],[3,4]])
    \end{block}

    \begin{columns}[T]
        \begin{column}{0.50\textwidth}
            \begin{block}{Code Context}
                \small
                <code>\\
                import numpy as np\\
                a = np.array([[1,2],[3,4]])\\
                </code>\\
                result = $\ldots$ \# put solution in this variable\\
                BEGIN SOLUTION\\
                <code>
            \end{block}
        \end{column}
        \begin{column}{0.45\textwidth}
            \begin{block}{Unit Test}
                \small
                def test(result, ans):\\
                \ \ \ \ assert\_array\_equal(result, ans)\\
                \ \ \ \ return 1
            \end{block}
        \end{column}
    \end{columns}
\end{frame}

\begin{frame}{Evalution Metric}
    \begin{columns}[T] % align columns
        \begin{column}{.5\textwidth}
            The SelfEvolve model employs the `pass@1' metric, which is the ratio of correctly solved problems to total problems, \textit{with a few refinement steps}.
        \end{column}%
        \begin{column}{.5\textwidth}
            In this study, the CoT-SelfEvolve model is allowed to try maximum 5 attepmts to solve a problem, thus, the metric is `pass@5'.
        \end{column}%
    \end{columns}
\end{frame}
