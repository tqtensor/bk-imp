\appendix
\clearpage
\section{Prompts}
\subsection{Auto-CoT Prompt Generator (1)}
\label{sec:initial_cot}
\begin{verbatim}
You are a helpful Chain-of-Thought expert named as CoT-Guru that can understand the reasoning behind programming problems and provide step-by-step guidance to solve them. You have the ability to analyze code and generate a series of suggestions that guide others to reason and solve programming problems effectively. By leveraging your knowledge and expertise, you can assist users in understanding complex programming concepts and help them develop their problem-solving skills. Your goal is to empower users to think critically and logically about programming problems, enabling them to become better programmers.

Given the problem description with the code, and one or multiple StackOverflow posts, you need to learn from the comments to generate step-by-step suggestions that help another agent (CoT-SelfEvolve) to solve the problem.

The given problem is: {problem_description}.
The StackOverflow post with supportive comments is: {post}.
Please generate a series of suggestions or questions that guide CoT-SelfEvolve to reason and to solve the problem step-by-step.
Here are some suggestions:

Suggestion 1: [You should ...]
Suggestion 2: [, then ...]
Suggestion 3: [, then ...]
Final suggestion: [, and finally, ...]
\end{verbatim}

\subsection{Example of generated CoT prompt (1)}
\begin{verbatim}
Sure! Here are some suggestions to guide CoT-SelfEvolve in solving the problem step-by-step:

Suggestion 1: First, let's understand the problem. The given DataFrame has rows sorted by the "Type" column in ascending order. You want to shuffle the rows according to a given list. Is that correct?

Suggestion 2: To shuffle the rows, you can use the `iloc` function in pandas. It allows you to select rows based on their integer position. You can pass the desired order of rows as a list to the `iloc` function. Try using the `iloc` function with the given list [2, 4, 0, 3, 1, 5] and see what happens.

Suggestion 3: If the `iloc` function doesn't give you the desired result, you can try using the `reindex` function in pandas. The `reindex` function allows you to change the order of rows based on a given index. You can pass the desired order of rows as a list to the `reindex` function. Try using the `reindex` function with the given list [2, 4, 0, 3, 1, 5] and see if it gives you the desired result.

Final suggestion: If none of the above suggestions work, you can try using the `sample` function in pandas. The `sample` function allows you to randomly sample rows from a DataFrame. You can pass the number of rows you want to sample and set the `replace` parameter to `False` to ensure that the sampled rows are unique. Try using the `sample` function with the number of rows equal to the length of the DataFrame and see if it gives you the desired result.

Remember to assign the result to the `result` variable.
\end{verbatim}

\clearpage
\subsection{Initial Code Generator}
\label{sec:initial_code_generator}
\begin{verbatim}
Given the problem description with the code, you need to fulfill the task by writing the code that solves the problem.
The problem is: {problem_description}.
Your duty is to solve the problem described above by writing the code that solves the problem. You will replace the code inside the `[insert]` block with your code as following code context:
```python
{code_context}
```

Inside the context, you can see which libraries will be used, the input/output format, and the expected behavior of the code. And most imporantly, how your code will be tested. Please do not import any additional libraries as they have been provided in the context. Make sure your code is correct and complete to solve the problem.

To support you in solving the problem, here are the Chain-of-Thought reasoning suggestions, you should follow these suggestions one by one, to use them as a guide for your internal reasoning process to solve the problem.
{cot_suggestion}
\end{verbatim}

\subsection{Auto-CoT Prompt Generator (2)}
\label{sec:correction_cot}
\begin{verbatim}
Given the problem description with the code, and the code generated by another agent (CoT-SelfEvolve) together with the feedback from the system, you need to generate step-by-step Chain-of-Thought reasoning to help the CoT-SelfEvolve to solve the problem by himself.
The given problem is: {problem_description}.

In the previous attempt, CoT-SelfEvolve generated the following code:
GENERATED_CODE:
```
{generated_code}
```

And it received the following feedback:
FEEDBACK:
```
{feedback}
```

In general, the steps to reason about the problem are:
- Step 1: understand the problem, what does it require?
- Step 2: analyze the GENERATED_CODE vs problem, what is the problem? does it related with the defined problem?
- Step 3: analyze the FEEDBACK, what is the error message? what is the expected output?

Please help the CoT-SelfEvolve agent by providing step-by-step guidance to solve the problem. DO NOT attempt to solve the problem directly. Remember that you are helping another agent to solve the problem, not solving the problem directly.
\end{verbatim}

\subsection{Example of generator CoT prompt (2)}
\begin{verbatim}
Sure! Let's go through the steps to reason about the problem and help the CoT-SelfEvolve agent.

Step 1: Understand the problem
The problem requires shuffling the order of the DataFrame's rows according to a given list.

Step 2: Analyze the GENERATED_CODE vs problem
The GENERATED_CODE is incomplete and needs to be modified to solve the problem. The `df` and `List` variables are loaded from the pickle file, but the code inside the `[insert]` block is missing. The goal is to shuffle the rows of the DataFrame `df` according to the list `List`.

Step 3: Analyze the FEEDBACK
The feedback indicates that there is a NameError because the variable `result` is not defined. The expected output is not mentioned in the feedback.

Now that we have analyzed the problem and the code, let's provide step-by-step guidance to solve the problem.

Step 1: Shuffle the DataFrame
To shuffle the DataFrame `df` according to the list `List`, you can use the `iloc` function of pandas DataFrame. The `iloc` function allows you to select rows based on their integer positions. You can pass the list `List` as the argument to `iloc` to select the rows in the desired order.

Step 2: Modify the GENERATED_CODE
Replace the `[insert]` block in the GENERATED_CODE with the code to shuffle the DataFrame. Assign the shuffled DataFrame to the variable `result`.

Step 3: Save the result
After shuffling the DataFrame, save the `result` to the pickle file.

Remember, the goal is to guide the CoT-SelfEvolve agent to solve the problem by himself.
\end{verbatim}

\subsection{Correction Code Generator}
\label{sec:correction_code_generator}
\begin{verbatim}
In the previous attempt, you generated the following code:
GENERATED_CODE:
```
{generated_code}
```
However, the system has given you the following instruction:
FEEDBACK:
```
{feedback}
```
Please comply with the instruction and generate the code accordingly.

To support you in solving the problem, here are the Chain-of-Thought reasoning suggestions, you should follow these suggestions one by one, to use them as a guide for your internal reasoning process to solve the problem.
{cot_suggestion}
\end{verbatim}
