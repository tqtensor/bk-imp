\documentclass{article}
\author{}
\usepackage[utf8]{inputenc}
\usepackage{amsmath}
\usepackage{amssymb}
\usepackage{amsmath,amsthm,amssymb,scrextend}
\usepackage{fancyhdr}
\usepackage{graphicx}
\usepackage{hyperref}
\begin{document}

\title{Homework 3 \- Dr.\ Nguyen An Khuong}
\maketitle
\section{Student Information}
\begin{itemize}
    \item \bf{Student Name:} Tang Quoc Thai
    \item \bf{Student ID:} 2270376
\end{itemize}

\section{Exercises}
\subsection{Exercise 1}
The row averages of $A_{0}$ are 3 and 0, therefore, $A = \begin{bmatrix} 2 & 1 & 0 & -1 & -2 \\ -1 & 1 & 0 & 1 & -1 \end{bmatrix}$.
And $S = \frac{AA^{T}}{4} = \begin{bmatrix}
    2.5 & 0 \\
    0 & 1
\end{bmatrix}$.
The eigenvalues of $S$ are $\lambda_{1} = 2.5 \text{ and } \lambda_{2} = 1$.
\\
A vertical line is closer to the five points in the columns of A than any other line through the origin (0, 0).


\subsection{Exercise 3}
$A_{0} = \begin{bmatrix}
    1 & 2 & 3 \\
    5 & 2 & 2
\end{bmatrix}$ has row averages of 2 and 3, therefore, $A = \begin{bmatrix}
    -1 & 0 & 1 \\
    2 & -1 & -1
\end{bmatrix}$.
Then $S = \frac{AA^{T}}{2} = \frac{1}{2} \begin{bmatrix}
    2 & -3 \\
    -3 & 6
\end{bmatrix}$.\\
The eigenvalues of $S$ are roots of the equation $(2 - \lambda)(6 - \lambda) - 9 = 0$. $\lambda_{1} = 4 + \sqrt{13}$ and $\lambda_{2} = 4 - \sqrt{13}$ and $\sigma_{1} = \sqrt{\lambda_{1}} = \sqrt{4 + \sqrt{13}}$ and $\sigma_{2} = \sqrt{\lambda_{2}} = \sqrt{4 - \sqrt{13}}$.

\subsection{Exercise 5}
Assume $D = \begin{bmatrix}
    x & & \\
    & y & \\
    & & z
\end{bmatrix}$, and $DS = \begin{bmatrix}
    x & & \\
    & y & \\
    & & z
\end{bmatrix}
\begin{bmatrix}
    4 & 2 & 0 \\
    2 & 4 & 1 \\
    0 & 1 & 1
\end{bmatrix} =
\begin{bmatrix}
    4x & 2x & 0 \\
    2y & 4y & y \\
    0 & z & z
\end{bmatrix}$\\
Therefore, $DSD = \begin{bmatrix}
    4x & 2x & 0 \\
    2y & 4y & y \\
    0 & z & z
\end{bmatrix}
\begin{bmatrix}
    x & & \\
    & y & \\
    & & z
\end{bmatrix} =
\begin{bmatrix}
    4x^{2} & 2xy & 0 \\
    2xy & 4y^{2} & yz \\
    0 & yz & z^{2}
\end{bmatrix}$\\
Since this matrix has 1's on the diagonal so $x = 1/2, y = 1/2, z = 1$, $D = \begin{bmatrix}
    1/2 & & \\
    & 1/2 & \\
    & & 1
\end{bmatrix}$.

\subsection{Exercise 7}
Subtract the average grade of each course (each row in $A_{0}$) from the 10 grades in that row to create the centered matrix $A$. The sample covariance matrix is then calculated as $S = \frac{1}{9} AA^{T}$. The primary eigenvector of the $5 \times 5$ matrix $S$ provides the weights for the 5 courses to generate the `eigencourse'. This represents the course with the highest variance in grades, hence containing the most information.\\
If a course awards an A to every student, it has zero variance and is therefore not factored into the eigencourse. The goal is to find the course with the most information, not necessarily the highest grade.
\end{document}
