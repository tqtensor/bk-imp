\documentclass{article}
\author{}
\usepackage[utf8]{inputenc}
\usepackage{amsmath}
\usepackage{amssymb}
\usepackage{amsmath,amsthm,amssymb,scrextend}
\usepackage{fancyhdr}
\begin{document}

\title{Homework 1 - Dr. Nguyen An Khuong}
\maketitle
\section{Team members}
\begin{itemize}
    \item Ngo Trieu Long
    \item Tang Quoc Thai
    \item Nguyen Trung Thuan
    \item Do Vo Hoang Hung
    \item Nguyen Dinh Khuong
\end{itemize}

\section{Exercises}
\subsection{Exercise 1}
Assuming we have a generic 2x2 matrix as below.
\[
    \begin{bmatrix}
        a & b \\
        c & d
    \end{bmatrix}
\]

To find the eigenvalues of the matrix \( A = \begin{bmatrix} a & b \\ c & d \end{bmatrix} \), we solve the characteristic equation obtained from the determinant of \( A - \lambda I \), where \( I \) is the identity matrix:

\[
    \text{det}(A - \lambda I) = \begin{vmatrix} a - \lambda & b \\ c & d - \lambda \end{vmatrix} = (\lambda^2 - (a+d)\lambda + (ad-bc))
\]

So, finding the eigenvalues is equivalent to solving the following equation: \( \lambda^2 - (a+d)\lambda + (ad-bc) = 0 \).

Once we have the eigenvalues \( \lambda_1, \lambda_2 \), we can find the corresponding eigenvectors \( x_1, x_n \) by solving the system of linear equations \( (A - \lambda_i I)x_i = 0 \) for each \( i \):

\[
    (A - \lambda_i I)x_i = \begin{bmatrix} a - \lambda_i & b \\ c & d - \lambda_i \end{bmatrix} \begin{bmatrix} x_{i1} \\ x_{i2} \end{bmatrix} = \begin{bmatrix} 0 \\ 0 \end{bmatrix}
\]

If \(c\) is not zero, and the eigenvalues are \(\lambda_1\) and \(\lambda_2\), the corresponding eigenvectors can be expressed as:

For \(\lambda_1\):
\[
    x_1 = \begin{bmatrix} \lambda_1 - d \\ c \end{bmatrix}
\]

For \(\lambda_2\):
\[
    x_2 = \begin{bmatrix} \lambda_2 - d \\ c \end{bmatrix}
\]

If \( b \) is not zero, and the eigenvalues are \( \lambda_1 \) and \( \lambda_2 \), the corresponding eigenvectors can be expressed as:

For \( \lambda_1 \):
\[
    x_1 = \begin{bmatrix} b \\ \lambda_1 - a \end{bmatrix}
\]

For \( \lambda_2 \):
\[
    x_2 = \begin{bmatrix} b \\ \lambda_2 - a \end{bmatrix}
\]

If both \( b \) and \( c \) are zero, and the eigenvalues are \( \lambda_1 \) and \( \lambda_2 \), the corresponding eigenvectors are:

For \( \lambda_1 \):
\[
    x_1 = \begin{bmatrix} 1 \\ 0 \end{bmatrix}
\]

For \( \lambda_2 \):
\[
    x_2 = \begin{bmatrix} 0 \\ 1 \end{bmatrix}
\]

Finally, the formula for an inverse matrix is as follows:

\[
    A^{-1} = \frac{1}{ad - bc} \begin{bmatrix} d & -b \\ -c & a \end{bmatrix}
\]

\subsubsection{Exercise 1.a}
Applying the formulas above.

- For \( A = \begin{bmatrix} 1 & 2 \\ 0 & 3 \end{bmatrix} \), \( \lambda^2 - 4\lambda + 3 = 0 \)

\[
    \begin{aligned}
        \lambda_1 & = 1
        \lambda_2 & = 3
    \end{aligned}
\]

Since b is not zero so \( X = \begin{bmatrix} 2 & 2 \\ 0 & 2 \end{bmatrix} = \begin{bmatrix} 1 & 1 \\ 0 & 1 \end{bmatrix} \)

\[
    X^{-1} = \frac{1}{(1)(1) - (1)(0)} \begin{bmatrix} 1 & -1 \\ 0 & 1 \end{bmatrix} = \begin{bmatrix} 1 & -1 \\ 0 & 1 \end{bmatrix}
\]

- For \( A = \begin{bmatrix} 1 & 1 \\ 3 & 3 \end{bmatrix} \), \( \lambda^2 - 4\lambda = 0 \)

\[
    \begin{aligned}
        \lambda_1 & = 0
        \lambda_2 & = 4
    \end{aligned}
\]

Since b is not zero so \( X = \begin{bmatrix} 1 & 1 \\ -1 & 3 \end{bmatrix} \)

\[
    X^{-1} = \frac{1}{(1)(3) - (-1)(1)} \begin{bmatrix} 3 & -1 \\ 1 & 1 \end{bmatrix} = \begin{bmatrix} 3/4 & -1/4 \\ 1/4 & 1/4 \end{bmatrix}
\]

\subsubsection{Exercise 1.b}
\[
    A^{3} = X\Lambda^{3}X^{-1}
\]

\[ A^{-1} = X \Lambda^{-1} X^{-1} \], because \[ A^{-1} A = X \Lambda^{-1} X^{-1} X \Lambda X^{-1} = X \Lambda^{-1} I \Lambda X^{-1} = X I X^{-1} = I \]

\subsection{Exercise 3}

We can replace the fact that \( A = X \Lambda X^{-1} \) and \( I = X X^{-1} \) into \( A + 2I \), we will have:

\[
    A + 2I = X \Lambda X^{-1} + 2 X X^{-1}
    = X (\Lambda + 2I) X^{-1}
\]

So, the eigenvalue matrix is \( \Lambda + 2I \), the eigenvector matrix is \( X \).

\subsection{Exercise 4}
(a) False. Even if all eigenvectors of \(A\) are linearly independent, some still can correspond to eigenvalue \(\lambda = 0\). Because the determinant of the matrix \(A\) is the product of its eigenvalues, we have \(\det(A) = 0\), which implies that \(A\) cannot be invertible.

(b) True. Because we have \(n\) independent eigenvectors that can form a basis \(\mathbb{R}^n\) for matrix \(A \in \mathbb{R}^{n \times n}\), it follows that \(A\) is diagonalizable.

(c) True. Because all the columns of \(X\) are independent, \(X\) is a full-rank matrix, which implies that \(X\) can be inverted.

(d) False. Not enough to draw a conclusion, because there are invertible matrices can not be diagonalize.

\subsection{Exercise 6}
\[
    A=
    \begin{bmatrix}
        4 & 0 \\
        1 & 2
    \end{bmatrix}
\]
$\det(\lambda I-A) = 0$
$\Rightarrow \det\left(
    \begin{bmatrix}
            1 & 0 \\
            0 & 1
        \end{bmatrix} -
    \begin{bmatrix}
            4 & 0 \\
            1 & 2
        \end{bmatrix}\right) =
    \begin{bmatrix}
        \lambda - 4 & 0          \\
        0           & \lambda -2
    \end{bmatrix} = 0.$
So there are two roots:
$\lambda = 2$ or $\lambda = 4$.

Find eigenvectors for $\lambda = 4$. Find all vectors $X \neq 0$ such that $AX = 4X$.

We have:
$(4I - A) X = 0 \Rightarrow X =
    \begin{bmatrix}
        4 & 0 \\
        1 & 0
    \end{bmatrix}
    \begin{bmatrix}
        x \\
        y
    \end{bmatrix} = \begin{bmatrix}
        0 \\
        0
    \end{bmatrix} \Rightarrow y = 0 \Rightarrow x = \begin{bmatrix}
        2i \\
        i
    \end{bmatrix}$


Find eigenvectors for $\lambda = 2$. Find all vectors $X \neq 0$ such that $AX = 2X$.

We have:
$(2I - A) X = 0 \Rightarrow X =
    \begin{bmatrix}
        2 & 0 \\
        1 & 0
    \end{bmatrix}
    \begin{bmatrix}
        x \\
        y
    \end{bmatrix} = \begin{bmatrix}
        0 \\
        0
    \end{bmatrix} \Rightarrow y = 0 \Rightarrow x = \begin{bmatrix}
        i \\
        0
    \end{bmatrix}$

$\Rightarrow S = \begin{bmatrix}
        2 & 0 \\
        1 & 1
    \end{bmatrix}$.
Having that $A = X \Lambda X^{-1} = X^{-1} \Lambda (X^{-1})^{-1}.$
We can conclude that all the matrices are diagonal. $A^{-1}$ are the inverse matrices $X^{-1}$ of matrix $X$.

\subsection{Exercise 11}
(a) True. $\det(A) = \lambda_1 \lambda_2 \lambda_3 = 2 \times 2 \times 5 \neq 0 \Rightarrow$ can be invertible.

(b) False. Because 2 eigenvalues have the same values $\Rightarrow$ we lack of independent eigenvector to form the basis $R^n$

(c) True.

\subsection{Exercise 12}
(a) False. Multiples of $(1,4)$ eigenvector could respond to a nonzero eigenvalue $\Rightarrow$ $\det(A) \neq 0$ $\Rightarrow$ $A$ is invertible.

(b) True. If not, we would have distinct (independent) eigenvectors.

(c) True, since there are not enough independent eigenvectors $\Rightarrow$ eigenvector matrix $X$ is not invertible $\Rightarrow$ $A$ is not diagonalizable.

\subsection{Exercise 26}
If matrix a has distinct eigenvalues $\Rightarrow$ each corresponding eigenvector is linear independent  $\Rightarrow$ we can have n linearly independent eigenvectors.
+ In the case of $\lambda \neq 0$ $\Rightarrow$ $\lambda$ might be repeated eigenvalues $\Rightarrow$ matrix A might have too few independent eigenvector to from basis R$^n$
\end{document}
